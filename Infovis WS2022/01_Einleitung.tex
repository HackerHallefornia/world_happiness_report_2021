\section{Einleitung}

Zufriedenheit und Glück sind Themen die in der Forschung mehr und mehr an Bedeutung gewinnen. \cite{veenhoven2012cross}
Wie lassen sich diese Erfahrungen oder Emotionen überhaupt quantifizieren? Der \textit{World Happiness Report} ist wohl eines der umfassendsten Forschungsprojekte welches sich mit diesem Thema und dem Glück auf der Welt am ausführlichsten Auseinandersetzt. Die beteiligten Forscher erstellen jedes Jahr einen umfassenden Bericht über alle Länder über die sie Daten erfassen können und stellen ein Happiness-ranking auf, also wie glücklich und zufrieden die verschiedenen Länder im jeweiligen Jahr sind. Diese Daten bilden die Grundlage dieser Arbeit. \\

In den letzten Jahren hat sich die Lebenszufriedenheit in vielen Teilen der Welt verbessert, insbesondere aufgrund der wirtschaftlichen Entwicklung und der Fortschritte in der Medizin und im Bildungswesen. \cite{lozano2018measuring} Gleichzeitig gibt es jedoch auch viele Herausforderungen, die Lebenszufriedenheit beeinträchtigen können, wie zum Beispiel Arbeitslosigkeit, Armut, gesundheitliche Probleme und soziale Isolation. Gerade durch die COVID-19 Pandemie wurden viele Länder und deren Lebensqualität und Zufriedenheit beeinträchtigt.\cite{prime2020risk} \cite{krendl2021impact} Auch damit beschäftigt sich der World Happiness Report 2021. \cite{helliwell_world_2021} Die hier aufgeworfenen Fragen sind jedoch zu umfangreich für die Visualisierungen, daher wird Umfang der behandelten Themen auf die Daten des Berichtes beschränkt. \\

Eine visuelle Aufbereitung des World Happiness Reports könnte für eine Vielzahl von Personen und Organisationen von Nutzen sein, die an den Faktoren interessiert sind, die die Lebenszufriedenheit in verschiedenen Ländern beeinflussen. Dazu könnten zum Beispiel Regierungen, Unternehmen, Nichtregierungsorganisationen (NGO), Wissenschaftler, Journalisten und interessierte Bürger gehören. \\

Da der Bericht selber eine sehr ausführliche Beschreibung und Diskussion der Daten bietet, ist es Ziel dieser Arbeit einen schnelleren und intuitiven Zugriff auf die Daten und die enthaltenen Informationen zu bieten. Dabei sollen sich die Nutzer folgende Fragen mit den interaktiven Visualisierungen beantworten können. \\

\begin{itemize}
    \item Wie sind Ländergruppen im Happiness Score und erfassten Größen einzuordnen? Sind Zusammenhänge zu erkennen? 
    \item Was sind die Werte für ein bestimmtes Land? Wo befindet es sich im Vergleich zu den anderen Ländern? 
    \item Wie konstant sind die Werte über die letzten 10-20 Jahre. Entwickeln sich Länder stetig weiter oder machen einige auch Rückschritte?
\end{itemize}

\subsection{Anwendungshintergrund}
Das Gebiet der Lebenszufriedenheit ist ein wichtiger Bereich der Psychologie und Sozialwissenschaften, der sich mit dem subjektiven Empfinden von Glück, Zufriedenheit und Wohlbefinden befasst. \cite{frawley2015happiness} Die Lebenszufriedenheit wird oft als ein wichtiger Indikator für das allgemeine Wohlbefinden und die Qualität des Lebens betrachtet, da sie mit einer Vielzahl von positiven Größen wie besserer Gesundheit, längerer Lebenserwartung und höherer Leistungsfähigkeit in Beziehung gebracht wird. \cite{steptoe2019happiness}
\\

Es gibt viele Faktoren, die die Lebenszufriedenheit beeinflussen, wie zum Beispiel persönliche Eigenschaften, soziale Beziehungen, Einkommen, Gesundheit, Umgebung und Lebensstil. Diese Faktoren können interagieren und sich gegenseitig beeinflussen, wodurch die Lebenszufriedenheit komplex und dynamisch ist. \cite{fernandez2001contribution}
\\

Der World Happiness Report beschäftigt sich seit Jahren intensiv mit diesem Thema. Diese jährlich veröffentlichte Studie wird von der United Nations Sustainable Development Solutions Network (SDSN) erstellt und bewertet die Lebenszufriedenheit in verschiedenen Ländern auf der Grundlage von Daten, die aus verschiedenen Quellen stammen, einschließlich Umfragen und Statistiken zu verschiedenen Indikatoren wie Einkommen, Gesundheit, Arbeitsbedingungen und sozialem Zusammenhalt. \cite{helliwell_world_2021} Der World Happiness Report bietet eine umfassende Perspektive auf die Lebenszufriedenheit auf globaler Ebene und kann dazu beitragen, das Verständnis der Faktoren, die die Lebenszufriedenheit beeinflussen, zu vertiefen und Maßnahmen zur Verbesserung der Lebenszufriedenheit zu identifizieren. 


\subsection{Zielgruppen}

Da der Inhalt der Daten wie bereits erwähnt für eine Vielzahl von Personengruppen von Interesse ist, ist es schwierig hier eine konkrete Zielgruppe, oder Zielgruppen, festzulegen. Da wie bereits erwähnt die wissenschaftlichen Interessen und detaillierten Ausführungen bereits im World Happiness Report selber gefunden werden können soll hier der Fokus auf eine nicht-wissenschaftliche Gruppe oder Gruppen gelegt werden. \\

Hier kommen ein paar Personengruppen in Frage. Eine gute Zielgruppe sind Lehrer welche ihren Schülern im Politik oder Geographie Unterricht Informationen über Lebenszufriedenheit in der Welt vermitteln möchten, hier könnten die Visualisierung als Unterrichtsunterstützung und Startpunkt von Diskussionen dienen. Auch einfach Zusammenhänge können gezeigt und diskutiert werden. \\

Eine weitere Zielgruppe sind NGOs die sich mit humanitäre Hilfe und ähnlichem Auseinandersetzten. Die Visualisierungen können ihnen dabei helfen Leuten auf einfache und Verständliche Art und Weise Ungleichheiten in der Welt in vielen verschieden Faktoren zu vermitteln und so mehr Aufmerksamkeit schaffen. 

\subsection{Überblick und Beiträge}
Die durch die Visualisierungen dargestellten Daten sind zweigeteilt. Die Durchschnitte der Jahre 2018-2020 sind die Basis aller Visualisierungen die keine Zeitliche Komponente enthalten. Sie sollen einen Ist-Zustand darstellen. Der zweite Datensatz sind die historischen Daten, diese bilden die Grundlage für die Zeitreihen Darstellung. \\

Die drei verwendeten Visualisierungen tragen unterschiedlich zu der Beantwortung der Nutzerfragen bei und vereinfachen allgemein die Zugänglichkeit der Daten. Der Scatterplot ermöglicht es Variablen und deren Korrelation übersichtlich darzustellen, zudem lassen sich hier Muster innerhalb von Ländergruppen identifizieren, da die Länder in Gruppen unterteilt und entsprechend Farblich hervorgehoben wurden. \\

Die zweite Visualisierung, der Polarplot gibt einen überblickt über die Variablen für ein Land. man gewinnt auf einem Blick einen Eindruck dafür wie ein Land aufgestellt ist in den verschiedenen Bereichen. Die letzte Visualisierung ist eine Zeitreihe mit leicht anderen gegebenen Variablen deren Erfassung teils bis zum Jahr 2006 zurückreichen. Hier lässt sich die Entwicklung von zwei Ländern gleichzeitig übersichtlich betrachten und vergleichen. \\
\section{Einleitung}

Zufriedenheit und Glück sind Themen die in der Forschung mehr und mehr an Bedeutung gewinnen. [Quelle]
Wie lassen sich diese Erfahrungen oder Emotionen überhaupt quantifizieren. Der World Happiness ist wohl eines der Umfassensten Forschungsprojekte welches sich mit diesem Thema und dem Glück auf der Welt am ausführlichsten Auseinandersetzt. Die Forscher erstellen jedes Jahr einen umfassenden bericht über alle Länder über die sie Daten erfassen können und stellen ein Happiness-ranking auf, also wie glücklich und zufrieden die verschiedenen Länder im jeweiligen Jahr sind. Diese Daten bilden die Grundlage dieser Arbeit. \\

In den letzten Jahren hat sich die Lebenszufriedenheit in vielen Teilen der Welt verbessert, insbesondere aufgrund der wirtschaftlichen Entwicklung und der Fortschritte in der Medizin und im Bildungswesen. Gleichzeitig gibt es jedoch auch viele Herausforderungen, die Lebenszufriedenheit beeinträchtigen können, wie zum Beispiel Arbeitslosigkeit, Armut, gesundheitliche Probleme und soziale Isolation. Gerade durch die COVID-19 Pandemie wurden viele Länder und deren Lebensqualität und Zufriedenheit beeinträchtigt. Auch damit beschäftigt sich der World Happiness Report 2021.  [quelle] \\

Eine visuelle Aufbereitung des World Happiness Reports könnte für eine Vielzahl von Personen und Organisationen von Nutzen sein, die an den Faktoren interessiert sind, die die Lebenszufriedenheit in verschiedenen Ländern beeinflussen. Dazu könnten zum Beispiel Regierungen, Unternehmen, Nichtregierungsorganisationen, Wissenschaftler, Journalisten und interessierte Bürger gehören. \\

Da der Bericht selber eine sehr ausführliche Beschreibung und Diskussion der Daten bietet, ist es Ziel dieser Arbeit einen schnelleren und intuitiven Zugriff auf die Daten und die enthaltenen Informationen zu bieten. Dabei sollen sich die Nutzer folgende Fragen mit den interaktiven Visualisierungen beantworten können. \\

\begin{itemize}
    \item Wie Zufrieden sind die Länder im Vergleich zueinander?
    \item Sind Zusammenhänge (Korrelation) zu erkennen zwischen der Zufriedenheit und erhobenen Variablen?
    \item Wie konstant sind die Werte über die letzen 10-20 Jahre. Entwickeln sich Länder stetig weiter oder machen einige auch Rückschritte?
\end{itemize}

\subsection{Anwendungshintergrund}
Das Gebiet der Lebenszufriedenheit ist ein wichtiger Bereich der Psychologie und Sozialwissenschaften, der sich mit dem subjektiven Empfinden von Glück, Zufriedenheit und Wohlbefinden befasst. Die Lebenszufriedenheit wird oft als ein wichtiger Indikator für das allgemeine Wohlbefinden und die Qualität des Lebens betrachtet, da sie mit einer Vielzahl von positiven Größen wie besserer Gesundheit, längerer Lebenserwartung und höherer Leistungsfähigkeit in Beziehung gebracht wird.
\\
Es gibt viele Faktoren, die die Lebenszufriedenheit beeinflussen, wie zum Beispiel persönliche Eigenschaften, soziale Beziehungen, Einkommen, Gesundheit, Umgebung und Lebensstil. Diese Faktoren können interagieren und sich gegenseitig beeinflussen, wodurch die Lebenszufriedenheit komplex und dynamisch ist.
\\
Der World Happiness Report beschäftigt sich seit Jahren instensiv mit diesem Thema. Diese jährlich veröffentlichte Studie wird von der United Nations Sustainable Development Solutions Network (SDSN) erstellt und bewertet die Lebenszufriedenheit in verschiedenen Ländern auf der Grundlage von Daten, die aus verschiedenen Quellen stammen, einschließlich Umfragen und Statistiken zu verschiedenen Indikatoren wie Einkommen, Gesundheit, Arbeitsbedingungen und sozialem Zusammenhalt. Der World Happiness Report bietet eine umfassende Perspektive auf die Lebenszufriedenheit auf globaler Ebene und kann dazu beitragen, das Verständnis der Faktoren, die die Lebenszufriedenheit beeinflussen, zu vertiefen und Maßnahmen zur Verbesserung der Lebenszufriedenheit zu identifizieren. 


\subsection{Zielgruppen}

Beschreiben sie die Personengruppe oder Personengruppen, die das von ihnen benannte Anwendungsproblem lösen möchte. Auf welches Vorwissen können sie in dieser Gruppen von Anwenderinnen aufbauen? Welche Informations"-bedürf"-nisse werden durch die Visualisierungen adressiert?
\subsection{Überblick und Beiträge}
In diesem Abschnitt geben sie einen kurzen Überblick über die Daten und verwendeten Visualisierungen. Dann benennen sie die Beiträge ihres Projekts. Diese Beiträge müssen sie in den hinteren Teilen des Berichts genauer ausführen und belegen.

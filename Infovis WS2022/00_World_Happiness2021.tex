\documentclass[usegeometry=true]{scrartcl}
\usepackage[ngerman]{babel}
\usepackage[T1]{fontenc}
\usepackage{lmodern}
\usepackage[utf8]{inputenc}
\usepackage{hyperref}
\usepackage{amssymb}
\usepackage{csquotes}
\usepackage{graphicx}
\usepackage[table,xcdraw]{xcolor}
\usepackage[left=2cm, right=2cm, top=2cm, bottom=2cm, bindingoffset=1cm, includeheadfoot]{geometry}
%Zeilenabstand bitte nicht ändern
\usepackage[onehalfspacing]{setspace}
\setlength{\parindent}{0em}
\usepackage[backend=biber,style=numeric,]{biblatex}\addbibresource{literatur.bib}

\begin{document}
% ----------------------------------------------------------------------------
\subject{Projektbericht zum Modul Information Retrieval und Visualisierung Sommersemester 2022}
\title{World Happiness Report 2021}
%\subtitle{Untertitel}% optional
\author{Johannes Boldt Matr.Nr.:}% obligatorisch
\date{}
\maketitle% verwendet die zuvor gemachte Angaben zur Gestaltung eines Titels
\pagenumbering{gobble}
% ----------------------------------------------------------------------------
\clearpage


% Inhaltsverzeichnis:
\tableofcontents
\clearpage
% ----------------------------------------------------------------------------
% Gliederung und Text:
\pagenumbering{arabic}

\section{Einleitung}

Zufriedenheit und Glück sind Themen die in der Forschung mehr und mehr an Bedeutung gewinnen. [Quelle]
Wie lassen sich diese Erfahrungen oder Emotionen überhaupt quantifizieren. Der World Happiness ist wohl eines der Umfassensten Forschungsprojekte welches sich mit diesem Thema und dem Glück auf der Welt am ausführlichsten Auseinandersetzt. Die Forscher erstellen jedes Jahr einen umfassenden bericht über alle Länder über die sie Daten erfassen können und stellen ein Happiness-ranking auf, also wie glücklich und zufrieden die verschiedenen Länder im jeweiligen Jahr sind. Diese Daten bilden die Grundlage dieser Arbeit. \\

In den letzten Jahren hat sich die Lebenszufriedenheit in vielen Teilen der Welt verbessert, insbesondere aufgrund der wirtschaftlichen Entwicklung und der Fortschritte in der Medizin und im Bildungswesen. Gleichzeitig gibt es jedoch auch viele Herausforderungen, die Lebenszufriedenheit beeinträchtigen können, wie zum Beispiel Arbeitslosigkeit, Armut, gesundheitliche Probleme und soziale Isolation. Gerade durch die COVID-19 Pandemie wurden viele Länder und deren Lebensqualität und Zufriedenheit beeinträchtigt. Auch damit beschäftigt sich der World Happiness Report 2021.  [quelle] \\

Eine visuelle Aufbereitung des World Happiness Reports könnte für eine Vielzahl von Personen und Organisationen von Nutzen sein, die an den Faktoren interessiert sind, die die Lebenszufriedenheit in verschiedenen Ländern beeinflussen. Dazu könnten zum Beispiel Regierungen, Unternehmen, Nichtregierungsorganisationen, Wissenschaftler, Journalisten und interessierte Bürger gehören. \\

Da der Bericht selber eine sehr ausführliche Beschreibung und Diskussion der Daten bietet, ist es Ziel dieser Arbeit einen schnelleren und intuitiven Zugriff auf die Daten und die enthaltenen Informationen zu bieten. Dabei sollen sich die Nutzer folgende Fragen mit den interaktiven Visualisierungen beantworten können. \\

\begin{itemize}
    \item Wie Zufrieden sind die Länder im Vergleich zueinander?
    \item Sind Zusammenhänge (Korrelation) zu erkennen zwischen der Zufriedenheit und erhobenen Variablen?
    \item Wie konstant sind die Werte über die letzen 10-20 Jahre. Entwickeln sich Länder stetig weiter oder machen einige auch Rückschritte?
\end{itemize}

\subsection{Anwendungshintergrund}
Das Gebiet der Lebenszufriedenheit ist ein wichtiger Bereich der Psychologie und Sozialwissenschaften, der sich mit dem subjektiven Empfinden von Glück, Zufriedenheit und Wohlbefinden befasst. Die Lebenszufriedenheit wird oft als ein wichtiger Indikator für das allgemeine Wohlbefinden und die Qualität des Lebens betrachtet, da sie mit einer Vielzahl von positiven Größen wie besserer Gesundheit, längerer Lebenserwartung und höherer Leistungsfähigkeit in Beziehung gebracht wird.
\\

Es gibt viele Faktoren, die die Lebenszufriedenheit beeinflussen, wie zum Beispiel persönliche Eigenschaften, soziale Beziehungen, Einkommen, Gesundheit, Umgebung und Lebensstil. Diese Faktoren können interagieren und sich gegenseitig beeinflussen, wodurch die Lebenszufriedenheit komplex und dynamisch ist.
\\

Der World Happiness Report beschäftigt sich seit Jahren instensiv mit diesem Thema. Diese jährlich veröffentlichte Studie wird von der United Nations Sustainable Development Solutions Network (SDSN) erstellt und bewertet die Lebenszufriedenheit in verschiedenen Ländern auf der Grundlage von Daten, die aus verschiedenen Quellen stammen, einschließlich Umfragen und Statistiken zu verschiedenen Indikatoren wie Einkommen, Gesundheit, Arbeitsbedingungen und sozialem Zusammenhalt. Der World Happiness Report bietet eine umfassende Perspektive auf die Lebenszufriedenheit auf globaler Ebene und kann dazu beitragen, das Verständnis der Faktoren, die die Lebenszufriedenheit beeinflussen, zu vertiefen und Maßnahmen zur Verbesserung der Lebenszufriedenheit zu identifizieren. 


\subsection{Zielgruppen}

Die Zielgruppe dieser Arbeit sind alle Personen die sich für Lebenszufriedenheit in der Welt interessieren, ausgenommen solcher die hierzu selber Forschung betreiben. Von viel Vorwissen ist daher nicht auszugehen. Daher müssen die Visualisierungen klar vermitteln können was dargestellt wird. Komplexe Darstellungen sind nicht angebracht. \\
Die Personen dieser Gruppe möchten mehr über Lebenszufriedenheit in der Welt erfahren und sich einen Eindurck vermitteln lassen wie diese verteilt ist. Diese Bedürfnisse sollen die Visualisierungen bedienen. 

\subsection{Überblick und Beiträge}
Die durch die Visualisierungen dargestellten Daten sind zweigeteilt. Die Durchschnitte der Jahre 2018-2020 sind die Basis aller Visualisierungen die keine Zeitliche Komponente enthalten. Sie sollen einen Ist-Zustand darstellen. Der zweite Datensatz sind die historischen Daten, diese bilden die Grundlage für die Zeitreihen Darstellung. \\

Die drei verwendeten Visualisierungen tragen unterschiedlich zu der Beantwortung der Nutzerfragen bei und vereinfachen allgemein die Zugönglichkeit der Daten. Der Scatterplot ermöglicht es Variablen und deren Korrelation übersichtlich darzustellen, zudem lassen sich hier Muster innerhalb von Ländergruppen identifizieren, da die Länder in Gruppen unterteilt und entsprechend Farblich hervorgehoben wurden. \\

Die zweite Visualisierung, der Polarplot gibt einen überblickt über die Variablen für ein Land. man gewinnt auf einem Blick einen Eindruck dafür wie ein Land aufgestellt ist in den verschiedenen Bereichen. Die letzte Visualisierung ist eine Zeitreihe mit leicht anderen gegebenen Variablen deren Erfassung teils bis zum Jahr 2006 zurückreichen. Hier lässt sich die Entwicklung von zwei Ländern gleichzeitig übersichtlich betrachten und vergleichen.

\section{Daten}

Die verwendeten Daten sind aus dem Kaggle-Datensatz \textit{World Happiness Report 2021}, die enthaltenen Daten wurden durch den World Happiness Report erfasst. Ajaypal Singh, der Ersteller des Kaggle Datensatzes hat keine Veränderungen an den gegebenen Daten vorgenommen, das Format wurde nur zu \textit{.csv} geändert. Man erhält die gleichen Daten wenn man sie von der \textit{World Happiness Report 2021} Seite herunterlädt. \cite{helliwell_world_2021}  \\

Die bereitgestellten Daten beeinhalten zwei separate Datensätze. Einen Datensatz mit historischen Daten, die für manche Länder bis 2005 zurückreichen, mit 9 gemessenen Größen für jedes Jahr. Und einen Datensatz der einen Vergleich der Länder für den Zeitraum 2018 - 2020 erstellt mit zusätzlichen Größen. Diese sind die ausgerechneten Faktoren für die gemessenen Größen und wie viel stark diese die Zufriedenheit in dem jeweiligen Land beeinflussen. \\

\begin{table}[h]
\centering
\resizebox{\textwidth}{!}{%
\begin{tabular}{|l|l|}
\hline
\rowcolor[HTML]{EFEFEF} 
Spaltenname &
  Beschreibung \\ \hline
Country name &
  Name des Landes \\ \hline
Regional indicator &
  Region zu welcher das Land gehört \\ \hline
Ladder score &
  \begin{tabular}[c]{@{}l@{}}Der Zufriedenheits Wert.\end{tabular} \\ \hline
Logged GDP per capita &
  Log umgewandeltes Bruttoinslandseinkommen pro Kopf \\ \hline
Social support &
  \begin{tabular}[c]{@{}l@{}}Nationaler Durchschnitt auf die Binäre Frage: {[}If you\\ were in trouble, do you have relatives or friends you can count on to help you\\ whenever you need them, or not?{]}\end{tabular} \\ \hline
Healthy life expectancy &
  Daten von der WHO über die gesunde Lebenserwartung im Land. \\ \hline
Freedom to make life choices &
  \begin{tabular}[c]{@{}l@{}}Nationaler Durchschnitt auf die Frage: {[}Are you satisfied or dissatisfied \\
    with your freedom to choose what you do with your life?{]} \end{tabular}\\ \hline
Generosity &
  \begin{tabular}[c]{@{}l@{}}Nationaler Durchschnitt auf die Frage, {[}Have you donated money to a charity in the \\ past month?{]}, regressiert auf das Bruttoinlandsprodukt pro Kopf\end{tabular} \\ 
\hline 
Perceptions of corruption &
  \begin{tabular}[c]{@{}l@{}}Nationaler Durchschnitt auf die Fragen: {[}Is corruption widespread throughout\\ the government or not{]} und {[}Is corruption widespread within businesses or not?{]}\end{tabular} \\ \hline
\end{tabular}%
}
\caption{Größen aus dem ersten Datensatz.}
\label{Tab:dat_1}
\end{table}

Tabelle \ref{Tab:dat_1} enthält die Größen des ersten Datensatzes und deren grobe Beschreibungen. Der \textit{Ladder Score} ist der Zufriedenheitswert oder Happiness Score. Der Kernwert des World Happiness Reports. Er wird \textit{Ladder Score} genannt aufgrund der Frage mit der er in Umfragen erfasst wird. Diese lautet unübersetzt: \textit{Please imagine a ladder, with steps numbered from 0 at the
bottom to 10 at the top. The top of the ladder represents the best possible life
for you and the bottom of the ladder represents the worst possible life for you.
On which step of the ladder would you say you personally feel you stand at this
time?"} \cite{helliwell_world_2021}. Der nationale Durchschnitt aus Antworten auf dies Frage bildet dann den Happiness Score. \\

Allerdings wurden einiger Variablen weggelassen. Jede der Variablen nach \textit{Ladder score} hatte eine weiter Instanz mit einer statischen Verechnung dieser um eine \textit{Explained by} Größe zu bilden. Diese stellen dar wie stark diese Größe wohl den erreichten Ladder Score erklärt. Da diese Variablen eher komplex sind und bereits im \textit{World Happiness Report}  ausführlich dargestellt werden, wurden sie nicht verwendet. Die verwendeten Fragen für die Variablenerfassung wurden nicht aus dem Englischen Übersetzt um deren Bedeutung nicht zu verzerren. Weiter Statistische Größen die ausgelassen wurden, sind der \textit{Standard Error}, \textit{lower whisker} und \textit{upper whisker}. \\

Der zweite Datensatz enthält alle der in Tabelle \ref{Tab:dat_1} enthaltenen numerischen Größen und zusätzlich noch die Größen \textit{positive affect} und \textit{negative affect}. Diese stellen Durchschnittswerte auf eine weiter Befragung dar. Für \textit{positive affect} ist sind es drei Fragen. Ob man innerhalb des Tages häufig Glücksgefühle erlebt, häufig lacht oder häufig zufrieden ist. Dies werden getrennt gestellt dann zusammen gefasst es ergibt sich eine Zahl zwischen 0 und 1. Das gleiche Prinzip gilt für \textit{negative affect}. Hier sind die erfragten Emotionen Sorge, Trauer und Wut. 

\subsection{Datenvorverarbeitung}

Die Daten wurden zuerst auf Vollständigkeit geprüft. Die Zusammengefassten Daten waren bereits vollständigkeit und wiesen keine fehlenden Einträge auf. Bei den Zeitreihendaten gab es einige fehlenden Einträge. Immer wieder wurden in einem Jahr nur wenige Werte eingetragen, oder es fehlten ganze Jahre. Als erster Schritt wurden alle Länder entfernt, welche nnicht in den Zusammenfassungsdaten vorhanden waren um eine Vergleichsbasis zu schaffen. Anschließend wurden noch unvollständige Jahreseinträge entfernt. Fehlende Einträge kompletter Jahre wurden nicht ergänzt und die entsprechenden Länder aber auch nicht aus der Liste genommen. Dies hätte den Datensatz sonst stark dezimiert. Die fehlenden Jahresdaten werden im Scatterplot einfach interpoliert. \\

Die überflüssigen Spalten aus der \texttt{world-happiness-report-2021.csv} wurden mit einem addon der IDE VSCode entfernt. Das waren die Daten Felder, welche \textit{Explained by:} enhielten, dies wurden absichtlich ausgelassen, wie zuvor erwähnt . Die Verarbeitung der Zeitreihendaten fand durch einen kurzen Pythonskript statt. Die Einträge mit fehlenden Werte wurden entfernt und so eine gekürzt Version geschaffen. Eine alternative wäre es gewesen die Daten hier zu interpolieren, davon wurde abgesehen, da innerhalb der Zeitreihenvisualisierung ohnehin zwischen Datenpunkten interpoliert wird. Fehlt ein Eintrag in einem Jahr wird direkt eine Linie zum nächsten Jahr gezeichnet. 

\subsection{Datenbereitstellung}

Nach der Datenaufbereitung wurden die Daten in das Github Repository hochgeladen, welches auch die ELM Visualisierung enthält. Da es sich hier um zwei Datensätze handelt müssen auch zwei HTTP Get request gestellt werden. Dies wurde durch eine zusätzliche Command Message gelöst welche aufgerufen wird sobald der erste Datensatz erfolgreich geladen wurde. So werden die Daten in Sequenz geladen. 



\section{Visualisierungen}
\subsection{Analyse der Anwendungsaufgaben}
Analysieren sie die konkreten Anwendungsaufgaben, die die Lösung des Zielproblems durch die Anwender:innen bearbeitet werden müssen. 
Welche sinnvollen mentale Modelle helfen den Personen bei der Bearbeitung. 
%Welche Visualisierungen helfen den Personen, die die Software verwenden, sinnvolle mentale Modelle aufzubauen. 
Sind diese mentalen Modelle für sie notwendig, um die Aufgaben lösen zu können? Gehen sie bei ihrer Argumentation von den Anwendungsaufgaben aus und kommen sie dann zu den mentalen Modellen, deren Aufbau durch Visualisierungen unterstützt wird. 
\subsection{Anforderungen an die Visualisierungen}
Leiten sie Anforderungen an das Design der Visualisierungen ab, die sich durch ihre Analyse des Zielproblems ergeben.
\subsection{Präsentation der Visualisierungen}
Präsentieren sie die visuelle Abbildungen und Kodierungen der Daten und Interaktionsmöglichkeiten. 
Sie müssen  begründen, warum und wie gut ihre Designentscheidungen die erstellten Anforderungen erfüllen. 
Weiterhin müssen sie begründen, warum die gewählte visuelle Kodierung der Daten für das zulösenden Problem passend ist.
Typische Argumente würden hier auf Wahrnehmungsprinzipien und Theorie über Informationsvisualisierung verweisen. 
Die besten Begründungen diskutieren explizit die konkrete Auswahl der Visualisierungen im Kontext von mehreren verschiedenen Alternativen. 
Machen sie hier nicht den Fehler, einfach nur Visualisierung aus den vorgegebenen Bereichen zu diskutieren, weil das in der Regel nicht sinnvoll ist.
Wenn sie sich für einen Scatterplot entschieden haben, ist ein Zeitreihendiagramm in der Regel keine Alternative.
Diskutieren sie also nicht einfach Zeitreihendiagramme, weil sie in den Anforderungenen an das Projekt neben Scatterplots stehen, sondern suchen sie nach echten alternativen Visualisierungen, die zum Aufbau eines vergleichbaren mentalen Modells führen. 
Diskutieren sie die Expressivität und die Effektivität der einzelnen Visualisierungen. 

Die eben beschriebenen Präsentationen und Begründungen sollen für jede der drei folgenden Visualisierungen durchgeführt werden. 
\subsubsection{Visualisierung Eins}
\subsubsection{Visualisierung Zwei}
\subsubsection{Visualisierung Drei}

\subsection{Interaktion}
Die präsentierten Visualisierungstechniken müssen interaktiv zu einer Anwendung verknüpft werden.
Die Interaktion mit einer Visualisierung soll in den anderen Visualisierungen zu einer Änderung führen. 
Erklären sie die möglichen Interaktionen mit den einzelnen Visualisierungen und die möglichen Verknüpfungen zwischen ihnen. Begründen Sie warum die konkreten Interaktionen umgesetzt wurden und welche Zwecke für die Anwenderinnen mit ihnen unterstützt werden. Begründen sie ebenfalls warum sie andere Interaktionsmöglichkeiten nicht umgesetzt haben. Wenn sie keine der geforderten Interaktionen umsetzen, erhalten Sie im gesamten Projekt deutlichen Punktabzug. 

\section{Implementierung}
Beschreiben Sie die Implementierung ihrer Visualisierungsanwendung in Elm. Stellen die Gliederung ihres Quellcodes vor. Haben Sie verschiedene Elm-Module erstellt. Was war aufwändig umzusetzen, was ließ sich mit dem vorhanden Code aus den Übungen relativ einfach umsetzen? 

Wie sieht die Elm-Datenstruktur für das Model aus, in dem die verschiedenen Zustände der Interaktion gespeichert werden können.

\section{Anwendungsfälle}
Präsentieren sie für jede der drei Visualisierungen einen sinnvollen Anwendungsfall in dem ein bestimmter Fakt, ein Muster oder die Abwesenheit eines Musters visuell festgestellt wird. Begründen sie warum dieser Anwendungsfall wichtig für die Zielgruppe der Anwenderinnen ist. Diskutieren sie weiterhin, ob die oben beschriebene Information auch mit anderen Visualisierungstechniken hätte gefunden werden können. Falls dies möglich wäre, vergleichen sie die den Aufwand und die Schwierigkeiten ihres Ansatzes und der Alternativen. 
\subsection{Anwendung Visualisierung Eins}
\subsection{Anwendung Visualisierung Zwei}
\subsection{Anwendung Visualisierung Drei}

\section{Verwandte Arbeiten}
Führen sie eine kurze Literatursuche in der wissenschaftlichen Literatur zu Informationsvisualisierung und Visual Analytics nach ähnlichen Anwendungen durch. Diskutieren sie mindestens zwei Artikel. Stellen sie Gemeinsamkeiten und Unterschiede dar.

\section{Zusammenfassung und Ausblick}
Fassen sie die Beiträge ihre Visualisierungsanwendung zusammen. Wo bietet sie für die Personen der Zielgruppe einen echten Mehrwert.

Was wären mögliche sinnvolle Erweiterungen, entweder auf der Ebene der Visualisierungen und/oder auf der Datenebene?

\section*{Anhang: Git-Historie}

\printbibliography

\end{document}


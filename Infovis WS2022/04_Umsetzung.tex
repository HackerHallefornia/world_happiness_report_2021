\section{Implementierung}
\subsection{Allgemeiner Aufbau}
Die Elm Applikation ist in vier verschiedene Dateien/Module aufgeteilt. \texttt{Main.elm} ist dabei die Datei in welcher die eigentliche Visualisierung kompiliert wird. 
Sie greift auf die Module \texttt{Displaydata.elm}, \texttt{Polarplot.elm} und \texttt{Timeseries.elm} zu. 
Dabei enthält \texttt{Displaydata.elm} die Datenverarbeitungs Funktionalität sowie die Funktionen für die erste Visualisierung, den Scatterplot. 
Die \texttt{Polarplot.elm} Datei enthält die Funktionen für den Polarplot 
und die \break \texttt{Timeseries.elm} jene für die Timeseries.  Zusätzlich besteht noch \texttt{My\_types.elm} Datei, dies enthält die definierten Typen welche von den Modulen geteilt werden. Sie sind ausgesondert um Importschleifen innerhalb der Module zu verhindern.\\

\begin{figure}[ht]
\centering
\begin{mdframed}[backgroundcolor=backcolour]
\begin{minted}[breaklines, linenos, tabsize=2]{Elm}

type Model 
  =  Failure
  | Loading
  | Loaded WorldHappData

type Msg
  = GotText (Result Http.Error String)
  | GotTsdata (Result Http.Error String)
  | Scatterplot_yaxis Int
  | Scatterplot_xaxis Int
  | Polarplot_country Int
  | Timeseries_1 Int
  | Timeseries_2 Int
  | Timeseries_cat Int
  | Click_Scat Int
  
\end{minted}
\end{mdframed}
    \caption{Typen Msg und Model mit denen die Visualisierung gesteuert wird}
    \label{fig:Msg_Model}
\end{figure}

In der \texttt{Main.elm} ist die Model View Update Struktur zu finden. 
Hier wird in der init Funktion der command fetchData aufgerufen welcher die erste HTTP Get anfrage an die Datenseite im Github Repository schickt. 
Wird diese erfolgreich abgerufen wird ein Model erstellt welches den Datentyp \textit{WorldHappData} hat. 
Hier werden sowohl die Daten für die Visualisierungen gespeichert, als auch die Einstellungen für diese. Da zu diesem Zeitpunkt erst einer von zwei Datensätzen geladen ist, wird für den zweiten zunächst ein leerer Default abgespeichert. \\

Diese Speicherung löst den zweiten command aus, fetchTsData, dieser stellt den zweiten HTTP get request, dieses Mal für die historischen Daten. Wenn auch diese geladen wurden, sind alle Daten für die Seite vorhanden. Im View wird nun die Visualisierung aufgerufen. Hierzu zählen einige Elemente aus dem Bulma Modul\cite{elm-bulma} welche Textboxen und Styling von HTML in Elm vereinfachen. Die Visualisierungen befinden sich alle in der viewHappiness Funktion. Diese nimmt die Daten und Einstellungen und generiert daraus die Visualisierungen. \\


Wie in  Abbildung.\ref{fig:Msg_Model} zu sehen gibt es zwei Msg Typen die jeweils einen Result HTTP erzeugen. Dies sind die beiden HTTP get Requests die wie vorher genannt zuerst ausgeführt werden. Die restlichen Typen sind die Messages welche die Visualisierungen aktualisiert wenn ein Nutzer mit ihnen interagiert. Dabei wird jeweils aus einem Dropdown ein Element auswählt welches die Msg auslöst und so ein Update erzwingt, hierbei wird ein Int mitgereicht welcher zu einer Änderung der Einstellungen der jeweiligen Visualisierung korrespondiert. Die Einstellung wird im WorldHappData Model aktualisiert. \\

\begin{figure}[ht]
\centering
\begin{mdframed}[backgroundcolor=backcolour]
\begin{minted}[breaklines, linenos, tabsize=2]{Elm}

type alias WorldHappData = 
    { data : List Country_2021,
      ts_data : List Ts_data,
      y_axis : String,
      x_axis : String,
      polar_country : String,
      line_1 : String,
      line_2 : String,
      ts_cat : String
    }
  
\end{minted}
\end{mdframed}
    \caption{WorldHappData, Datentyp des Models}
    \label{fig:Worldhapp}
\end{figure}

Hier in Abbildung.\ref{fig:Worldhapp} ist der definierte Datentype des Models abgebildet. Er enthält jegliche Zustände des Modells welche durch die \textit{Type Msg} in Abbildung.\ref{fig:Msg_Model} aktualisiert werden. \\

\newpage
\subsection{Scatterplot}

Der Scatterplot baut in großen Teilen auf die bereits in den Übungen verwendeten Elemente auf. Hier werden auch dynamisch skalierte Achsen und nach Klassen angepasste Punkte eingesetzt. Um die Ländergruppen identifizieren zu können wurde eine Legende über dem Scatterplot hinzugefügt, diese besteht aus Rechtecken aus dem Typed.Svg Modul und zugehörigem Text. Die Funktion selbst nimmt 6 Argumente entgegen. Zwei Stringlisten, das sind die Ländernamen für jeden Punkt und die Region zu welcher sie gehören. Zwei \texttt{List Float} welche die x und y Werte für die jeweilig ausgewählten Attribute enthalten und noch zwei String welche die zugehörigen x-Achsen und y-Achsen Labels enthalten. \\

Die größte Änderung und Schwierigkeit und Abweichung vom in der Übung erstellten Scatterplots ist die Klick-Interaktion. Nutzer können die einzelnen Punkt anklicken und so die anderen Plots beeinflussen. Hierfür wurde ein onClick Event aus dem \textbf{TypedSvg} Modul verwendet \cite{typedsvg}. Dieses greift auf die in Abbildung.\ref{fig:Msg_Model} dargestellten Msg Types zu und löst den Typ \textit{Click\_scat} aus. 
Dieser aktualisiert dann die Auswahl des Landes im Polarplot und die erste Landesauswahl im Zeitreihen Plot. 

\subsection{Polarplot}

Der Polarplot basiert nicht auf den Übungsaufgaben, sondern auf einer Beispielvisualisierung von Gampleman \cite{polarplot}. Diese musste stark modifiziert werden um die finale Darstellung zu erhalten. Gampleman bietet keine Funktion mit der sich Punkte einzeichnen lassen, allerdings bieten die gezeichneten Achsen auf dem Beispiel eine gute Grundlage hierfür. Mittels eines Winkels und einem skalierten y Wert lassen sich Koordinaten ansteuern. Dieses Prinzip wurde übernommen und sechs Achsen geschaffen die den Kreis in gleichgroße Teile teilen. \\

Die Polarplot Funktion nimmt eine Liste von 6 Floats und zeichnet für jeden von ihnen auf einer der Achsen einen Punkt. Die Reihenfolge der Punkte korrespondiert hierbei mit der Reihenfolge der Achsen. Jede der Achsen hat eine angepasste Skalierung auf den Zahlenraum der darzustellenden Größe. Daher gibt es auf ein Liste von Skalen auf die hier zurückgegriffen wird. 

\subsection{Zeitreihe}

Diese Darstellung war wohl mit am einfachsten Umzusetzen, auch hier ist die Basis eine Gampleman Visualisierung, diese musste jedoch nicht ganz so stark modifiziert werden. Es handelt sich hier um den \textit{Lineplot} von Gampleman.\cite{linechart} Diese implementiert bereits eine zeitlich geordnete X-Achse. Für die gegebenen Daten war es allerdings nicht notwendig die komplexität von Zeitbasierten Datentypen zu verwenden. Die Jahreszahlen als Float zu verwenden war hier ausreichend und vereinfachte auch die Umsetzung der Visualisierung. 
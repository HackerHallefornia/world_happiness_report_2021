\section{Zusammenfassung und Ausblick}

In dieser Arbeit wurden drei Visualisierungen über die Daten des World Happiness Reports 2021 vorgestellt, beschrieben und mit ähnlichen Arbeiten verglichen. Die Nutzer der Visualisierungen sollten durch diese die Möglichkeit bekommen möglichst einfach die Daten des World Happiness Reports endecken zu können. Dafür wurde davon ausgegangen, dass die Zielgruppe nur wenig Hintergrundwissen über komplexere Visualisierungen mitbringt. Daher wurden Visualisierungen gewählt welche möglichst einfach und intuitiv Daten vermitteln können. \\

Die drei gewählten Visualisierungen waren ein Scatterplot, ein Polarplot und eine einfach Zeitreihe. Der Scatterplot dient als Einstiegspunkt in die Daten, hier werden immer zwei vom Nutzer auswählbare Attribute dargestellt, die Länder bilden dann die Punkte, welche sich auf dem Scatterplot verteilen. Der Nutzer bekommt hier die Möglichkeit weiter mit den Daten zu interagieren indem er die Namen der Länder erfährt wenn er seine Maus auf die einzelnen Punkte bewegt. Angeklickte Punkte wählen das entsprechende Land dann auch für die folgenden Visualisierungen aus. So sind die Visualisierungen verknüpft und führen den Nutzer von der einen in die andere. Die Daten sind von der Menge für den Scatterplot genau richtig, die Menge an Ländern reicht aus um Muster identifizierbar zu machen ohne zu viel Überlagerungen zu schaffen. Polarplot und Zeitreihe ermöglichen noch einen tieferen Einblick in den Ist-Zustand und die Vergangenheit einzelner Länder. \\

Die Visualisierungen ermöglichen es den Nutzern wie gewünscht einfach Muster in den Daten zu erkennen und Details über einzelne Länder sowie deren historische Entwicklung zu ermöglichen. Die Visualisierungen anderer Arbeiten in diesem Gebiet verfolgen hier meist ähnliche Ziele verwenden jedoch meist etwas andere Visualisierungsmethoden. Diese Arbeit gliedert sich hier gut ein. Eine Besonderheit sind hier die Zeitreihen für jede enthaltene Größe, dies kam in ähnlichen Berichten nicht vor. Der größte Mehrwehr hierbei ist es spielerschen Zugang zu den Daten zu erhalten, was dazu einlädt sich tiefer mit den Daten auseinanderzusetzten.\\

Es gibt hier noch viel Potenzial um mehr der Daten aus dem World Happiness Report sichtbar zu machen. Die Chloroplethen Karte, wie sie auch in \textcite{bazurto2019} zu finden ist, wäre eine gute Möglichkeit um mehr Geographischen Kontext und Einordnung zu bieten. Das erleichtert es dem Nutzer die Länder die er betrachtet einordnen zu können. Auch die bestehenden Visualisierungen könnten noch aufgewertet werden. Der Scatterplot ist gut darin Muster zu zeigen, es lässt sich aber nicht ein einzelnes Land identifizieren. Daher wäre eine Option sinnvoll mit der man ein Land auswählen kann, welches immer hervorgehoben wird während der Scatterplot verändert wird. 
\section{Visualisierungen}
In diesem Kapitel werden die Anwendungsaufgaben die zu lösen sind analysiert und deren Lösung erklärt. Dabei werden zuerst die Anforderungen festgestellt und diese dann in Bezug auf die Präsentation und Interaktion der Visualisierungen angewandt.  

\subsection{Analyse}

Das Ziel der Visualisierungen ist es die \textit{World Happiness Report} Daten intuitiv und einfach zugänglich zu machen und es der Zielgruppe zu ermöglichen Länder und deren Ranking zu identifizieren. Auch einzelne Länder und deren Variablen sollen schnell zugänglich sein. Zudem sollen sie in der Lage sein Zusammenhänge in den Daten identifizieren zu können und die zeitliche Entwicklung der Daten zu verfolgen. \\

Hierfür ist es entscheidend, die einzelnen Anwendungsaufgaben durch verschiedene Visualisierungen abzudecken und möglichst klar die benötigten Informationen zu vermitteln. 
Am sinnvollsten ist es vermutlich mit einem Überblick in die Daten also einer allgemeineren Visualisierung zu beginnen um einen Eindruck für die Daten und deren Zusammenhänge zu vermitteln. \\

\subsection{Anforderungen}
Die Anforderungen an die Visualisierungen sind vielfältig. Sie werden fallen für jede der Zielprobleme leicht unterschiedlich aus. Allgemein ist die Anforderung einfach verständliche Visualierungen zu verwenden, welche wenig bis gar kein Hintergrundwissen über Datenvisualisierungsmethoden erfordern. Damit fallen bereits einige Methoden weg. \\

Um die Länder und deren erfasste Variablen miteinander Vergleichen zu können muss ein Auswahl von Variablen und Identifizierung der Länder möglich sein. Dies liegt daran das sich 6 Variablen kaum visuell deutlich gleichzeitig für alle Länder darstellen lassen. Daher ist es sinnvoller jeweils nur ein oder zwei Attribute darzustellen. Dies soll durch Nutzerinteraktion geschehen. \\

Um auch einen direkten Vergleich aller Attribute zu ermöglichen ist es sinnvoll die Möglichkeit zu bieten jeweils nur ein Land zu betrachten, für dieses jedoch alle Attribute zu zeigen. Hier muss auf einen Blick erkennbar sein wie die Attribute zueinander aufgestellt sind. Auch hier muss Nutzerinteraktion ermöglicht werden um alle Länder in dieser Weise einzeln zu betrachten. \\

